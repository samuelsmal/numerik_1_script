\chapter{Solutions of linear Systems, LU decomposition, Factorization} % (fold)
\label{cha:solutions_of_linear_systems_lu_decomposition_factorisation}

Given a square matrix $\mathbf{A}$ of dimension $N \times N$ and a column vector 
$\mathbf{b} \in \mathbb{R}^N$, we want to solve $\mathbf{A}x = \mathbf{b}$.

\begin{Matlab}
	x = A \ b
\end{Matlab}

\[ 
	\begin{pmatrix}
		a_{1,1}x_1 & a_{1,2}x_2 & \cdots & a_{1,N}x_N \\
		a_{2,1}x_1 & a_{2,2}x_2 & \cdots & a_{2,N}x_N \\
		\vdots  & \vdots  & \ddots & \vdots  \\
		a_{N,1}x_1 & a_{N,2}x_N & \cdots & a_{N,N}x_N
	\end{pmatrix}
	=
	\begin{pmatrix}
		b_1 \\
		b_2 \\
		\vdots \\
		b_N
	\end{pmatrix}
 \]

Example
\[\begin{matrix*}[r]
		I   &  2x_1 & + &   3x_2 & + &   4x_3 & + &  5x_4 & = &   5 \\
		II  &  6x_1 & + &  15x_2 & + &  19x_3 & + & 23x_4 & = &  30 \\
		III &  8x_1 & + &  42x_2 & + &  60x_3 & + & 70x_4 & = &  98 \\
		IV  & 12x_1 & + &  60x_2 & + &   1x_3 & + & 17x_4 & = & 144 \\
\end{matrix*}\]


\paragraph{Step 1}

Eliminate $x_1$ from $II$, $III$, $IV$, while using only the first row:
The $n$ in $a_{*}^{(n)}$ indicates the used matrix, beginning with $0$ for the 
first matrix and increasing by one for each elimination step.

\[\begin{matrix*}[r]
	\tilde{II}  & = & II  & - & \frac{a_{21}^{(1)}}{a_{11}^{(0)}} I & = & II  & - & 3I \\
	\tilde{III} & = & III & - & \frac{a_{31}^{(0)}}{a_{11}^{(0)}} I & = & III & - & 4I \\
	\tilde{IV}  & = & IV  & - & \frac{a_{41}^{(0)}}{a_{11}^{(0)}} I & = & IV  & - & 6I \\
\end{matrix*}\]

Saving the step:

\[
	l_{21} = \frac{a_{21}^{(1)}}{a_{11}^{(0)}},
	l_{31} = \frac{a_{31}^{(0)}}{a_{11}^{(0)}},
	l_{41} = \frac{a_{41}^{(0)}}{a_{11}^{(0)}}
\]


\[
	\Rightarrow
	A^{(1)}|b^{(1)} = 
	\left(\begin{array}{r r r r | r}
			2 &  3 &   4 &   5 &   5 \\
			0 &  6 &   7 &   8 &  15 \\
			0 & 30 &  44 &  50 &  78 \\
			0 & 42 & -23 & -13 & 114
	\end{array}\right)
\]


\paragraph{Step 2}

Eliminate $x_2$ from $III$, $IV$, while using only the second row of $A^{(1)}$:

\[\begin{matrix*}[r]
		\tilde{III} & = & III & - & \frac{a_{32}^{(1)}}{a_{22}^{(1)}} II & = & III & - & 5II \\
		\tilde{IV}  & = & IV  & - & \frac{a_{42}^{(1)}}{a_{22}^{(1)}} II & = & IV  & - & 7II \\
\end{matrix*} \]

Saving the step:

\[
	l_{32} = \frac{a_{32}^{(1)}}{a_{22}^{(1)}},
	l_{42} = \frac{a_{42}^{(1)}}{a_{22}^{(1)}}
\]

\[
	\Rightarrow
	A^{(2)}|b^{(2)} = 
	\left(\begin{array}{r r r r | r}
			2 & 3 &   4 &   5 &  5 \\
			0 & 6 &   7 &   8 & 15 \\
			0 & 0 &   9 &  10 &  3 \\
			0 & 0 & -72 & -65 &  8
	\end{array}\right)
\]

\paragraph{Step 3}
Repeating the step for the last row:

\[\begin{matrix*}[r]
	\tilde{IV}  & = & IV  & - & \frac{a_{43}^{(2)}}{a_{33}^{(2)}} & = & IV  & + & 8I \\
\end{matrix*} \]

\[
	\Rightarrow
	A^{(3)}|b^{(3)} = 
	\left(\begin{array}{r r r r | r}
			2 & 3 &  4 &  5 &  5 \\
			0 & 6 &  7 &  8 & 15 \\
			0 & 0 &  9 & 10 &  3 \\
			0 & 0 &  0 & 15 & 32
	\end{array}\right)
\]

Saving the step:

\[
	l_{43} = \frac{a_{43}^{(2)}}{a_{33}^{(2)}}
\]

Note the form of $A^{(3)}$: It's in a \textit{upper triangle} from, and therefore
denoted with an $U$. The corresponding matrix build with the saved steps $l_{nm}$ is in a
\textit{lower triangle} and therefore denoted an $L$.

\[L = \begin{pmatrix*}[r]
	1      &      0 &      0 & 0 \\
	l_{21} &      1 &      0 & 0 \\
	l_{31} & l_{32} &      1 & 0 \\
	l_{41} & l_{42} & l_{43} & 1
\end{pmatrix*} =
\begin{pmatrix*}[r]
	1 & 0 & 0 & 0 \\
	3 & 1 & 0 & 0 \\
	4 & 5 & 1 & 0 \\
	6 & 7 & -8 & 1	
\end{pmatrix*}
\]

The diagonal is set to one to allow the following decomposition of $A$:

\[
	A = LU, b = L \cdot b^{(3)}
\]

% END Example

\paragraph{In the general case:} Given $Ax = b$, $A$ being a nonsingular
\footnote{A square matrix $A$ is called nonsingular, invertible, or 
nondegenerate if there exists a matrix $B$ such that $AB$ = $BA$ = $\mathbf{1}$.
A matrix is singular iff (if and only if) it's determinat is 0.} matrix.

Any square matrix $A$ admits an $LUP$ factorization. If $A$ is invertible, 
then it admits an $LU$ (or $LDU$) factorization if and only if all its leading 
principal minors\footnote{In linear algebra, a minor of a matrix $A$ is the 
determinant of some smaller square matrix, cut down from $A$ by removing one or 
more of its rows or columns. Minors obtained by removing just one row and one 
column from square matrices (first minors) are required for calculating matrix 
cofactors, which in turn are useful for computing both the determinant and 
inverse of square matrices.} are nonsingular. If A is a singular matrix of rank $k$ , 
then it admits an $LU$ factorization if the first $k$ leading principal minors 
are nonsingular, although the converse is not true.

If a square, invertible matrix has an $LDU$ factorization, then it is unique. 
In that case, the $LU$ factorization is also unique if we require that the 
diagonal of $L$ (or $U$ ) consists of ones.

The LU decomposition can be viewed as the matrix form of Gaussian elimination. 
Computers usually solve square systems of linear equations using the LU 
decomposition, and it is also a key step when inverting a matrix, or 
computing the determinant of a matrix. The LU decomposition was introduced
by mathematician Alan Turing.

Without a proper ordering or permutations in the matrix, 
the factorization may fail to materialize. For example, it is easy to verify 
(by expanding the matrix multiplication) that $a_{11} = l_{11} u_{11}$. 
If $a_{11} = 0$, then at least one of $l_{11}$ and $u_{11}$ has to be zero, 
which implies either $L$ or $U$ is singular. This is impossible if $A$ is 
nonsingular. This is a procedural problem. It can be removed by simply 
reordering the rows of $A$ so that the first element of the permuted matrix is 
nonzero. The same problem in subsequent factorization steps can be removed the 
same way, see the basic procedure below.

\begin{algorithm}
\caption{LU decomposition}
\label{alg:lu-decomposition} 
\begin{algorithmic}
	\For{$k = 1 \dotsc d - 1$} \Comment{for all $x_d$}
		\For{$i = 1 \dotsc k$} \Comment{For ... ???}
			\For{$j = 1 \dotsc d $} \Comment{For all columns}
    			\LState $a_{ij}^{(k+1)} \gets a_{ij}^{(k)}$
    		\EndFor
    	\EndFor
	    \For{$i = k + 1 \dotsc d$} \Comment{These equations change}
	    	\LState $l_{ik} \gets \frac{a_{ik}^{(k)}}{a_{kk}^{(k)}}$
	    	\For{$j = 1 \dotsc k$} \Comment{Elimination of $x_k$}
	    		\LState $a_{ij}^{(k + 1)} \gets 0$
	    	\EndFor
	    	\For{$j = k + 1 \dotsc d$}
	    		\LState $a_{ij}^{(k+1)} \gets a_{ij}^{(k)} - l_{ik}a_{kj}^{(k)}$
	    	\EndFor
	    	\LState $b_i^{(k + 1)} \gets b_i^{(k)} - l_{ik}b_{k}^{(k)}$
	    \EndFor
    \EndFor
\end{algorithmic}
\end{algorithm}

\begin{algorithm}
\caption{Backward substitution}
\label{alg:backward-substitution} 
\begin{algorithmic}
	\State $x_d \gets \frac{c_d}{U_{dd}}$
	\For{$i \gets d-1, d-2, \dotsc 1$}
		\LState $\tilde{c_i} \gets c_i - \sum_{j = i + 1}^{d} U_{dj}x_j$
		\LState $x_i \gets \frac{\tilde{c_i}}{U_{ii}}$
    \EndFor
\end{algorithmic}
\end{algorithm}

\begin{Theorem}
	$A = LU, b = Lc$ \\
	The Gaussian Elimination process yields a factorization of $A$ in terms of a
	lower and upper triangular matrix.
\end{Theorem}

% END OF PHOTO 4

% chapter solutions_of_linear_systems_lu_decomposition_factorisation (end)
