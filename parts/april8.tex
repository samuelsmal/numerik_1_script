\chapter{Gaussian elimination with partial pivoting}

Example %% TODO: Make Example-Environment

\[ 
	\mathbf{A}x = \mathbf{b}
	=
	\begin{pmatrix}
		-2x_1 - 3x_2 + 14x_3 =  4 \\
		 2x_1 + 2x_2 -  3x_3 =  5 \\
		 4x_1 + 2x_2 -  2x_3 = 10
	\end{pmatrix}
\]

\paragraph{Note:} 
There exists a solution because the determinante is non-zero.

\paragraph{Step 1:}
Scan rows 1 - 3 looking for the equation with maximal coefficient in \(x_1\)
(in absolute value). Put this equation in the first row (interchange first with 
the selected one).

\[
	\mathbf{\tilde{A}^{(1)}} =
	\begin{pmatrix}
 		4 2 -2 \\
 		2 2 -3 \\
 		-2 -3 14      
	\end{pmatrix} \text{,}
	\mathbf{\tilde{b}^{(1)}} =
	\begin{pmatrix}
		10 \\
		 5 \\
		 4
	\end{pmatrix}
\]

Now we eliminate \(x_1\) from equations 2 \& 3.

\[
	l_{21} = \frac{2}{4} = 0.5; l_{31} = \frac{-2}{4} = -0.5
\]

(Notice how \(|l_{21}| \leq 1\) and \(|l_{31|} \leq 1\).)

After elimminatio we obtain:

\[
	\mathbf{A^{(2)}} =
	\begin{pmatrix}
 		4  2 -2 \\
 		0  1 -2 \\
 		0 -2 13      
	\end{pmatrix} \text{, }
	\mathbf{b}^{(1)} =
	\begin{pmatrix}
		10 \\
		 0 \\
		 9
	\end{pmatrix}
\]

\paragraph{Step 2:}
Look again for an equation with biggest coefficient in \(x_2\), among rows 2 -3 
(we don't look anymore at row 1). Interchange rows 2 and 3.

\[
	\mathbf{\tilde{A}^{(2)}} =
	\begin{pmatrix}
 		4  2 -2 \\
 		0 -2 13 \\
 		0  1 -2 		   
	\end{pmatrix} \text{, }
	\mathbf{\tilde{b}(2)} =
	\begin{pmatrix}
		10 \\
		 4 \\
		 5
	\end{pmatrix}
\]

Eliminate \(x_2\) from equation 3: \(l_{32} = \frac{1}{-2} = -0.5\) (Again
\(|l_{32}| \leq 1\)).

\[
	\mathbf{A^{(3)}} =
	\begin{pmatrix}
 		4  2 -2 \\
 		0 -2 13 \\
 		0  0 4.5      
	\end{pmatrix} \text{, }
	\mathbf{b}^{(3)} =
	\begin{pmatrix}
		10 \\
		 9 \\
		 4.5
	\end{pmatrix} = c \text{, }
	\mathbf{L} = \begin{pmatrix}
 		1 0 0 \\
 		0.5 1 0 \\
 		-0.5 -0.5 1   
	\end{pmatrix}
\]

Now \(\mathbf{A} \neq \mathbf{LU}\)!

\[
	\mathbf{LU} = 
	\begin{pmatrix}
 		4  2 -2 \\
 		-2 -3 14 \\
 		2 2 -3  
	\end{pmatrix} =
	\mathbf{PA}
\]

\(\mathbf{LU}\) being the permuation of rows on \(\mathbf{A}\).

\[
	\mathbf{P} = \mathbf{P_2 P_1} =
	\begin{pmatrix}
 		0 0 1 \\
 		1 0 0 \\
 		0 1 0
	\end{pmatrix}
\]

Which leads to:

\[
	\mathbf{A} = \mathbf{P^{-1}LU}
\]

\begin{Matlab}
	[L, U, P] = lu(A)
\end{Matlab}
% END Example

\paragraph{Definition} A permuatation matrix is a matrix with the same rows as
the identity matrix but in a differnet order. All elements are $0$ but $d$ (for
a $d \times d$ matrix) which are $1$ and there are no $1$'s in the same row nor
column.
% END DEFINITION

\paragraph{Theorem} For every non singular $d \times d$ matrix $\mathbf{A}$ there
exists a permuation matrix $\mathbf{P}$ such that $\mathbf{PA}$ admits a 
$\mathbf{LU}$ factorization.

\[
	\mathbf{LU} = \mathbf{LU} \text{(when solving a linear sytem} \text{)}
	 	\mathbf{Pb} = \mathbf{Lc}
\]
% END THEOREM

With the above process (see example) all multipliers have absolute value $\leq 1$.